\chapter{Conclusões} \label{chap:conclusion}

Este trabalho realizou um estudo de um problema clássico na transferência de calor, especificamente abordando a equação de calor unidimensional em coordenadas cilíndricas. Baseando-se em hipóteses simplificadoras da literatura, adotamos a abordagem de simulação numérica pelo método numérico das linhas (NUMOL), com uma discretização baseada na técnica de diferenças finitas. O contexto aplicado envolveu varetas de combustíveis nucleares, proporcionando uma perspectiva prática e relevante.

O estudo incluiu simulações numéricas para avaliar a convergência da malha numérica em relação às variáveis espaciais e temporais, visando compreender a estabilidade do modelo numérico. Observou-se que todos os modelos analisados apresentaram convergência dos resultados.

Em seguida, foram realizadas simulações para quatro cenários distintos de geração de calor, explorando o comportamento da transferência de calor sujeita a geração interna, condução no domínio e convecção na superfície da vareta. Os resultados confirmaram os padrões esperados nos perfis de temperatura, com picos na origem do domínio e deslocamento do gradiente para a extremidade da vareta.

O estudo abrangeu também a variação de constantes das fontes de geração de calor e o número de Biot. Em todos os casos, o aumento desses parâmetros resultou em valores de temperatura mais elevados, alinhando-se às expectativas prévias. Os resultados foram comparados e validados a partir daqueles presentes na referência, que utilizou o método híbrido da Transformada Integral Generalizada (GITT).

Todos os códigos desenvolvidos encontram-se disponíveis de forma aberta no repositório online https://github.com/thiagoneye/project-monograph.

Por fim, este trabalho não apenas proporcionou ideias valiosas sobre a transferência de calor, mas também demonstrou a aplicação eficaz de métodos numéricos para resolver problemas nesse campo, consolidando um aprendizado significativo.