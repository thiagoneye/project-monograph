\chapter{Resultados e Discussão} \label{chap:results}

\section{Refinamento de Malha}

Iniciamos nossa análise dos resultados com um estudo sobre o refinamento das malhas temporal e espacial. O objetivo é compreender como os resultados se comportam diante de variações nesses parâmetros, ou seja, a estabilidade da solução e, consequentemente, determinar o número de pontos na malha numérica para utilização.

De acordo com \citet{schiesser2009}, o ajuste do espaçamento da grade em diferentes partes ou em todo o domínio do problema é conhecido como refinamento "h". Essa nomenclatura é comumente usada na literatura de análise numérica, onde "h" é símbolo para o espaçamento da grade. O refinamento "h" busca melhorar a precisão da solução, refinando o espaçamento da grade com base em estimativas de erro de truncamento local ou outros parâmetros de refinamento. É importante destacar que não existe uma única combinação universalmente ideal de refinamento; a solução de problemas complexos frequentemente envolve tentativa e erro para encontrar o equilíbrio entre precisão e carga computacional. O objetivo final é alcançar um limite predefinido para o erro global com o mínimo esforço computacional necessário.

\subsection{Malha Temporal}

Para a presente análise, buscou-se analisar o máximo erro absoluto de uma malha com a equivalente mais refinada, a fim de se compreender a estabilidade das mesmas.

\begin{gather}
    \epsilon = \max | \Delta \theta | = \max | \theta(r,t_{n+1}) - \theta(r,t_{n}) |
\end{gather}

Através do estudo do refinamento da malha temporal, pode-se observar que, ao refinar a malha numérica, não foram identificados erros significativos, como registrado na Tabela \ref{tab:error_in_the_temporal_mesh}.

O resultado de erro nulo obtido é um artefato metodológico que se origina da natureza dos solvers de EDO adaptativo. Estes solvers não utilizam os pontos de tempo fornecidos pelo usuário para realizar seus cálculos. Em vez disso, eles determinam dinamicamente uma sequência de passos de tempo internos, muito menores e mais numerosos, com o objetivo de manter o erro de truncamento local abaixo de uma tolerância pré-definida (\textit{RelTol} e \textit{AbsTol}). 

Portanto, optou-se por utilizar uma malha temporal com 100 pontos, que se mostrou apropriada para a resolução do problema em questão.

\begin{table}[H]
    \centering
    \caption{Erro numérico no refino da malha temporal.}
    \begin{tabular}{ccc}
        \textbf{Pontos na Malha} & \textbf{Tamanho da Malha {[}s{]}} & \textbf{Erro} \\ \hline
        11     & 0,15     & - \\
        21     & 0,075    & 0 \\
        41     & 0,0375   & 0 \\
        81     & 0,0188   & 0 \\
        161    & 0,00938  & 0 \\
        321    & 0,00469  & 0 \\
        641    & 0,00234  & 0 \\
        1.281  & 0,00117  & 0 \\
        2.561  & 5,86E-04 & 0 \\
        5.121  & 2,93E-04 & 0 \\
        10.241 & 1,46E-04 & 0 \\
        20.481 & 7,32E-05 & 0 \\
        40.961 & 3,66E-05 & 0  
    \end{tabular}
    \source{Autor, 2023.}
    \label{tab:error_in_the_temporal_mesh}
\end{table}

%Para mais detalhes, o Apêndice \ref{att:temporal_mesh_convergence_tables} apresenta os resultados do refino de malha temporal através de tabelas de convergência.

\subsection{Malha Espacial}

%Através do estudo do refinamento da malha espacial, constatou-se que o solver não é capaz de gerar resultados para malhas contendo menos de 76 pontos, gerando o seguinte erro: "Incapaz de atender às tolerâncias de integração sem reduzir o tamanho do passo abaixo do menor valor permitido (2,775558e-17) no tempo t."

Para a presente análise, buscou-se analisar o máximo erro absoluto de uma malha com a equivalente mais refinada, a fim de se compreender a estabilidade das mesmas.

\begin{gather}
    \epsilon = \max | \Delta \theta | = \max | \theta(r_{n+1},t) - \theta(r_{n},t) |
\end{gather}

O método numérico implementado converge, o que valida a simulação. Isso é evidenciado pela diminuição consistente do ``Erro Máximo'' à medida que a malha espacial é refinada (o número de pontos $n_r$ aumenta e o passo $\Delta r$ diminui, como registrado na Tabela \ref{tab:error_in_the_spatial_mesh}.

\begin{table}[H]
    \centering
    \caption{Erro numérico no refino da malha espacial.}
    \begin{tabular}{ccc}
        \textbf{Pontos na   Malha} & \textbf{Tamanho da Malha} & \textbf{Erro} \\
        \hline
        76   & 0,0133     & -          \\
        151  & 0,00666000 & 0,06208713 \\
        301  & 0,00333000 & 0,03133610 \\
        601  & 0,00166500 & 0,01692421 \\
        1.201 & 0,00083250 & 0,00698768 \\
        2.401 & 0,00041625 & 0,00378524 \\
        4.801 & 0,00020813 & 0,00417690 \\
        9.601 & 0,00010406 & 0,00141474
    \end{tabular}
    \source{Autor, 2023.}
    \label{tab:error_in_the_spatial_mesh}
\end{table}

Observa-se um comportamento anômalo entre as malhas de 2.401 e 4.801 pontos, onde o erro aumenta ligeiramente (de 0,00378524 para 0,00417690). As causas mais prováveis são:

\begin{itemize}
    \item Interação entre Erros: O erro total é uma combinação do erro de discretização espacial (que está sendo reduzido) e do erro de discretização temporal (que está fixo). Quando o erro espacial se torna muito pequeno, o erro total passa a ser dominado pelo erro temporal ou pela tolerância do solver, e a sua diminuição pode não ser mais monotônica.
    \item Erro da Solução de Referência: O erro é calculado em relação à solução da malha seguinte, que também é uma aproximação. Pequenas oscilações podem ocorrer devido a esta característica do método de análise.
\end{itemize}

A tendência de convergência é retomada no refinamento seguinte, confirmando que se trata de uma oscilação local e não de uma divergência.

Portanto, optou-se por utilizar uma malha espacial com 9.601 pontos, que se mostrou apropriada para a resolução do problema em questão.

%Para mais detalhes, o Apêndice \ref{att:spatial_mesh_convergence_tables} apresenta os resultados do refino de malha espacial através de tabelas de convergência.

\section{Perfis de Transferência de Calor}

Nesta seção, discutiremos os resultados obtidos para quatro formas de geração de calor. Além disso, faremos uma comparação com os resultados apresentados por \citet{soares2017}, que por sua vez se baseou no trabalho de \citet{bhattacharya2001}. É relevante mencionar que os resultados de \citet{soares2017} foram obtidos digitalmente, utilizando um software para coletar pontos em curvas. Portanto, devemos considerar que alguns erros podem surgir durante esse processo.

Para todos os casos, inicialmente o valor do termo fonte de calor uniforme adimensional (\(G ^*\)) foi considerado igual a \(32,4\) e o número de \textit{Biot} foi considerado igual a \(15\), ambos seguindo a literatura.

\subsection{Primeira Forma de Geração de Calor}

A primeira forma de geração de calor foi proposta por \citet{bhattacharya2001} e referenciada por \citet{soares2017}, ela se dá pela equação:

\begin{gather}
    G ' = G ^* t
    \label{eq:first_form_of_heat}
\end{gather}

\begin{figure}[H]
    \centering
    \caption{Perfis de temperatura para primeira forma de geração de calor.}
    
    \begin{subfigure}{0.45\textwidth}
        \includegraphics[width=1\linewidth]{figures/results/Fig01.png} 
        \caption{perspectiva isométrica.}
    \end{subfigure}
    \begin{subfigure}{0.45\textwidth}
        \includegraphics[width=1\linewidth]{figures/results/Fig02.png}
        \caption{vista superior.}
    \end{subfigure}
    
    \source{Autor, 2023.}
    \label{fig:surface01}
\end{figure}

O perfil de temperatura apresentado na forma de superfície na Figura \ref{fig:surface01} apresenta a variação de temperatura (adimensionalizada) através do raio da vareta de combustível, considerando a primeira forma de geração de calor, conforme equação \ref{eq:first_form_of_heat}.

Ao longo de todo o período de tempo, a origem da vareta exibiu o valor máximo de temperatura, enquanto a extremidade manteve a temperatura mínima. Esse padrão era esperado, considerando a fonte de calor na origem, que se propaga ao longo da vareta em direção à extremidade e, por fim, a superfície da vareta apresentando um resfriamento convectivo.

Observou-se, ainda, que a origem da vareta demonstrou um comportamento com uma temperatura inicial elevada, passando por um processo de resfriamento, seguido de um subsequente aquecimento, algo que não era esperado.

\begin{figure}[H]
    \centering
    \caption{Comparativo de resultados com a literatura para primeira forma de geração de calor.}    
    \includegraphics[scale=0.5]{figures/results/Fig03.png}
    \source{Autor, 2023.}
    \label{fig:profile01}
\end{figure}

Resultados análogos à Figura \ref{fig:surface01} foram apresentados na Figura \ref{fig:profile01}, mas dessa vez na forma de curvas para quatro instantes de tempo. Os resultados foram comparados com os apresentados por \citet{soares2017}, em todos os casos houve concordância entre os perfis de temperatura, com eventuais divergências decorrentes do método de obtenção dos dados (conforme supradito) e/ou da qualidade da malha espacial.

\subsection{Segunda Forma de Geração de Calor}

A segunda forma de geração de calor foi proposta por \citet{bhattacharya2001} e referenciada por \citet{soares2017}, ela se dá pela equação:

\begin{gather}
    G ' = G ^* r ^2 e ^{c_3 t}
    \label{eq:second_form_of_heat}
\end{gather}

\begin{figure}[H]
    \centering
    \caption{Perfis de temperatura para segunda forma de geração de calor.}
    
    \begin{subfigure}{0.45\textwidth}
        \includegraphics[width=1\linewidth]{figures/results/Fig04.png} 
        \caption{perspectiva isométrica.}
    \end{subfigure}
    \begin{subfigure}{0.45\textwidth}
        \includegraphics[width=1\linewidth]{figures/results/Fig05.png}
        \caption{vista superior.}
    \end{subfigure}
    
    \source{Autor, 2023.}
    \label{fig:surface02}
\end{figure}

O perfil de temperatura apresentado na forma de superfície na Figura \ref{fig:surface02} apresenta o variação de temperatura (adimensionalizada) através do raio da vareta de combustível, considerando a segunda forma de geração de calor, conforme equação \ref{eq:second_form_of_heat}.

Para o presente caso, o intervalo de tempo foi reduzido para reproduzir o fenômeno de forma mais significativa.

Durante a maior parte do período de tempo a origem da vareta exibiu o valor máximo de temperatura, enquanto que a extremidade apresentou a temperatura mínima, porém com o passar do tempo (últimos instantes da simulação) houve uma transição do valor máximo da origem para a extremidade.

Observou-se, ainda, que que todos os pontos do raio demonstraram um comportamento de crescimento incessante de temperatura, com o passar do tempo, indicando que a geração de calor na origem superou o resfriamento na superfície.

\begin{figure}[H]
    \centering
    \caption{Comparativo de resultados com a literatura para segunda forma de geração de calor.}
    \includegraphics[scale=0.5]{figures/results/Fig06.png}
    \source{Autor, 2023.}
    \label{fig:profile02}
\end{figure}

Resultados análogos à Figura \ref{fig:surface02} foram apresentados na Figura \ref{fig:profile02}, mas dessa vez na forma de curvas para quatro instantes de tempos. Os resultados foram comparados com os apresentados por \citet{soares2017}, em todos os casos houve concordância entre os perfis de temperatura, com eventuais divergências decorrentes do método de obtenção dos dados (conforme supradito) e/ou da qualidade da malha espacial.

\subsection{Terceira Forma de Geração de Calor}

A terceira forma de geração de calor foi proposta por \citet{soares2017}, ela se dá pela equação:

\begin{gather}
    G ' = G ^* (1 + c _1 t)
    \label{eq:third_form_of_heat}
\end{gather}

Nela, o valor do coeficiente \(c_1\) foi considerado igual a \(1\), seguindo a literatura.

\begin{figure}[H]
    \centering
    \caption{Perfis de temperatura para terceira forma de geração de calor.}
    
    \begin{subfigure}{0.45\textwidth}
        \includegraphics[width=1\linewidth]{figures/results/Fig07.png} 
        \caption{perspectiva isométrica.}
    \end{subfigure}
    \begin{subfigure}{0.45\textwidth}
        \includegraphics[width=1\linewidth]{figures/results/Fig08.png}
        \caption{vista superior.}
    \end{subfigure}
    
    \source{Autor, 2023.}
    \label{fig:surface03}
\end{figure}

O perfil de temperatura apresentado na forma de superfície na Figura \ref{fig:surface03} apresenta o variação de temperatura (adimensionalizada) através do raio da vareta de combustível, considerando a terceira forma de geração de calor, conforme equação \ref{eq:third_form_of_heat}.

De forma análoga a primeira forma de geração de calor, ao longo de todo o período de tempo, a origem da vareta exibiu o valor máximo de temperatura, enquanto a extremidade manteve a temperatura mínima. Novamente, esse padrão era esperado, considerando a fonte de calor na origem, que se propaga ao longo da vareta em direção à extremidade e, por fim, a superfície da vareta apresentando um resfriamento convectivo.

Observou-se, ainda, que com o passar do tempo houve um aumento expressivo de temperatura na origem com propagação para a região da superfície, indicando que com o passar do tempo a geração de calor tendeu a superar o resfriamento na superfície.

\begin{figure}[H]
    \centering
    \caption{Comparativo de resultados com a literatura para terceira forma de geração de calor.}
    \includegraphics[scale=0.5]{figures/results/Fig09.png}
    \source{Autor, 2023.}
    \label{fig:profile03}
\end{figure}

Resultados análogos à Figura \ref{fig:surface03} foram apresentados na Figura \ref{fig:profile03}, mas dessa vez na forma de curvas para quatro instantes de tempos. Os resultados foram comparados com os apresentados por \citet{soares2017}, em todos os casos houve concordância entre os perfis de temperatura, com eventuais divergências decorrentes do método de obtenção dos dados (conforme supradito) e/ou da qualidade da malha espacial.

\begin{figure}[H]
    \centering
    \caption{Influência nos perfis de temperatura devido a variação no coeficiente \(c_1\) da terceira forma de geração de calor.}
    \includegraphics[scale=0.5]{figures/results/Fig10.png}
    \source{Autor, 2023.}
    \label{fig:influence_of_coefficient_c1}
\end{figure}

Em seguida, buscou-se estudar a influência do coeficiente \(c_1\) nos perfis de temperatura. Através da Figura \ref{fig:influence_of_coefficient_c1} pode-se observar que com o aumento do valor do coeficiente, houve um aumento dos valores de temperatura, algo esperado devido a expressão da geração de calor.

\begin{figure}[H]
    \centering
    \caption{Influência nos perfis de temperatura devido a variação no número de \(Bi\) na terceira forma de geração de calor.}
    \includegraphics[scale=0.5]{figures/results/Fig11.png}
    \source{Autor, 2023.}
    \label{fig:influence_of_Biot_on_third_heat_generation}
\end{figure}

Por fim, buscou-se estudar a influência do número de Biot nos perfis de temperatura. Através da Figura \ref{fig:influence_of_Biot_on_third_heat_generation} pode-se observar que com o aumento do número de Biot, houve um descréscimo dos valores de temperatura, algo esperado devido a interpretação do mesmo (razão entre o coeficiente de transferência convectiva de calor na superfície e a condutância específica).

\subsection{Quarta Forma de Geração de Calor}

A quarta forma de geração de calor foi proposta por \citet{soares2017}, ela se dá pela equação:

\begin{gather}
    G ' = G ^* (1 + c_2 r ^2) e ^{c_3 t}
    \label{eq:fourth_form_of_heat}
\end{gather}

Nela, os coeficientes \(c_2\) e \(c_3\) foram considerados ambos iguais a \(1\), seguindo a literatura.

\begin{figure}[H]
    \centering
    \caption{Perfis de temperatura para quarta forma de geração de calor.}
    
    \begin{subfigure}{0.45\textwidth}
        \includegraphics[width=1\linewidth]{figures/results/Fig12.png} 
        \caption{perspectiva isométrica.}
    \end{subfigure}
    \begin{subfigure}{0.45\textwidth}
        \includegraphics[width=1\linewidth]{figures/results/Fig13.png}
        \caption{vista superior.}
    \end{subfigure}
    
    \source{Autor, 2023.}
    \label{fig:surface04}
\end{figure}

O perfil de temperatura apresentado na forma de superfície na Figura \ref{fig:surface04} apresenta o variação de temperatura (adimensionalizada) através do raio da vareta de combustível, considerando a quarta forma de geração de calor, conforme equação \ref{eq:fourth_form_of_heat}.

De forma análoga a terceira forma de geração de calor, ao longo de todo o período de tempo, a origem da vareta exibiu o valor máximo de temperatura, enquanto a extremidade manteve a temperatura mínima. Novamente, esse padrão era esperado, considerando a fonte de calor na origem, que se propaga ao longo da vareta em direção à extremidade e, por fim, a superfície da vareta apresentando um resfriamento convectivo.

Observou-se, ainda, que com o passar do tempo houve um aumento expressivo de temperatura na origem com propagação para a região da superfície, indicando que com o passar do tempo a geração de calor tendeu a superar o resfriamento na superfície.

\begin{figure}[H]
    \centering
    \caption{Comparativo de resultados com a literatura para quarta forma de geração de calor.}
    \includegraphics[scale=0.5]{figures/results/Fig14.png}
    \source{Autor, 2023.}
    \label{fig:profile04}
\end{figure}

Resultados análogos à Figura \ref{fig:surface04} foram apresentados na Figura \ref{fig:profile04}, mas dessa vez na forma de curvas para quatro instantes de tempos. Os resultados foram comparados com os apresentados por \citet{soares2017}, em todos os casos houve concordância entre os perfis de temperatura, com eventuais divergências decorrentes do método de obtenção dos dados (conforme supradito) e/ou da qualidade da malha espacial.

\begin{figure}[H]
    \centering
    \caption{Influência nos perfis de temperatura devido a variação no número de \(Bi\) na quarta forma de geração de calor.}
    \includegraphics[scale=0.5]{figures/results/Fig15.png}
    \source{Autor, 2023.}
    \label{fig:influence_of_Biot_on_fourth_heat_generation}
\end{figure}

Por fim, buscou-se estudar a influência do número de Biot nos perfis de temperatura. Através da Figura \ref{fig:influence_of_Biot_on_fourth_heat_generation} pode-se observar que com o aumento do número de Biot, houve um descréscimo dos valores de temperatura, algo esperado devido a interpretação do mesmo (razão entre o coeficiente de transferência convectiva de calor na superfície e a condutância específica).
