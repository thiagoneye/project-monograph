\chapter*{Abstract}

The present work conducted a numerical study of the classic problem of one-dimensional heat transfer in cylindrical coordinates, considering its application to fuel rods in nuclear reactors. The study employed a numerical approach focused on the heat equation, utilizing the method of lines (NUMOL) with finite difference discretization. Throughout the investigation, mesh convergence simulations were performed to comprehend the solution's stability when refining temporal and spatial meshes. Moreover, comprehensive simulations were carried out for various heat generation scenarios, incorporating variations in fundamental variables of the physical phenomenon. This approach facilitated an understanding of heat transfer behavior, accounting for internal conduction and surface convection of the fuel rods. The validation of these simulations involved comparing the obtained results with existing literature, highlighting the convergence and robustness of the numerical approach. In conclusion, this study not only provided a valuable tool for a deeper understanding of heat transfer concepts but also offered unparalleled insights into the operation of nuclear reactors. The latter aspect holds significant importance in the current global scenario characterized by population growth, accelerated technological development, a growing emphasis on sustainability, and the need for diversification of the global energy matrix. The application of numerical methods to address challenges in this specific context emerged as an effective and relevant approach. \\

\textbf{Keywords:} Heat Transfer. Mesh Refinement. Method of Lines Numerical. Nuclear Fuel. Numerical Simulation. 