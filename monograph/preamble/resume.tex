\chapter*{Resumo}

O presente trabalho realizou um estudo numérico do problema clássico da transferência de calor unidimensional em coordenadas cilíndricas, considerando a aplicação em varetas combustíveis de reatores nucleares, por meio de uma abordagem numérica focada na equação de calor. Foi utilizado o método numérico das linhas (NUMOL) com discretização por diferenças finitas. Ao longo do estudo foram realizadas simulações de convergência da malha, buscando compreender a estabilidade da solução ao se realizar um refinamento das malhas temporal e espacial. Além disso, foram conduzidas simulações abrangentes para diferentes cenários de geração de calor, considerando variações em variáveis fundamentais do fenômeno físico. Esse enfoque permitiu uma compreensão do comportamento da transferência de calor, considerando a condução interna e a convecção na superfície das varetas de combustível. A validação dessas simulações foi realizada comparando os resultados obtidos com a literatura, evidenciando a convergência dos resultados e a robustez da abordagem numérica. Por fim, o estudo não apenas forneceu uma ferramenta valiosa para o aprofundamento de conceitos de transferência de calor, mas também ofereceu ideias com importância ímpar sobre o funcionamento dos reatores nucleares. Este último aspecto é de grande importância no cenário mundial atual, marcado pelo aumento populacional, desenvolvimento tecnológico acelerado, a crescente ênfase na sustentabilidade e a necessidade de diversificação da matriz energética global. A aplicação de métodos numéricos para solucionar desafios nesse contexto específico destacou-se como uma abordagem eficaz e relevante. \\

\textbf{Palavras-chaves:} Combustível Nuclear. Método Numérico das Linhas. Refino de Malha. Simulação Numérica. Transferência de Calor.